\begin{savequote}[75mm] 
Nulla facilisi. In vel sem. Morbi id urna in diam dignissim feugiat. Proin molestie tortor eu velit. Aliquam erat volutpat. Nullam ultrices, diam tempus vulputate egestas, eros pede varius leo.
\qauthor{Quoteauthor Lastname} 
\end{savequote}

\chapter{Role of crystals in fast timing}

\section{Time profiles}
%vasiliev

\section{The role of Cerenkov photons}
%Serve un calcolo delle soglie per i cristalli usati e i principali parametri

\section{Influence of rise and decay time of TOF-PET}
%discussion da seifert

\section{How to measure: methods}

\subsection{TCSPC}

\subsection{Excitation}
\subsection{Detection}

\section{Data analysis techniques}

\subsection{Accuracy and bias problems}
In time correlated single photon counting experiments the problem at hand is statistically speaking to estimate or more parameters (the lifetimes) from a dataset.

It is important to recall here the Rao-Cramer theorem 
%magari in appendice
that states that in general the variance of an estimate can not be smaller than a well-defined limit, that identifies the efficient estimator.

\begin{equation}
F_{hj} = \sum _{i}\frac{1}{y_{i}}\frac{\delta y_{i}}{\delta \alpha _{h}}\frac{\delta y_{i}}{\delta \alpha _{j}}
\end{equation}

\begin{equation}
P(n, \alpha _{1}, \alpha _{2},...) = \frac{N!}{n_{1}!...n_{k}!} p_{1}^{n_{1}}\cdot ... \cdot p_{k}^{n_{k}}
\end{equation}

\begin{equation}
(F^{multi})_{hj} = \sum _{i} \frac{1}{Np_{i}}\frac{\delta Np_{i}}{\delta \alpha _{h}}\frac{\delta Np_{i}}{\delta \alpha _{j}} = N\sum _{i}\frac{1}{p_{i}}\frac{\delta p_{i}}{\delta \alpha _{h}}\frac{\delta p_{i}}{\delta \alpha _{j}}
\end{equation}

\begin{equation}
var_{N}(\tau) = (F^{m})^{-1}=\frac{1}{N}[F^{m}(N=1)]^{-1}
\end{equation}

\begin{equation}
N > \frac{var_{1}(\tau)}{required variance (\tau)}
\end{equation}

%appendice sulla Poisson

\begin{equation}
p_{i} = \int _{\Delta T_{i}} f(t)dt
\end{equation}

\begin{equation}
f(t, \tau , T) = \frac{1}{\tau} exp(-t/tau)\frac{1}{1-exp(-T/tau)} 
\end{equation}

\begin{equation}
p_{i}(t, \tau, T) = \int _{(i-1)T/k} ^{iT/k} f(t, \tau, T)dt = frac{exp(\frac{T}{\tau k}) - 1}{1-exp(- t/\tau)} \cdot exp(-\frac{iT}{\tau k})
\end{equation}

\begin{equation}
f(t, \tau _{R}, \tau _{D}, T) = [exp(-t/tau _{D}) - exp (-t/tau _{R})] \frac{1}{\tau _{R}[exp(-T\tau _{R})-1] - \tau _{D}[exp(-T\tau _{D})-1]}  
\end{equation}


% qui il bias

\begin{equation}
P(m;\epsilon) = \epsilon ^{m} e ^{-\epsilon} / m!
\end{equation}

\begin{equation}
Event_{unbias} = U = P(1;\epsilon) = \epsilon e ^{-\epsilon}
\end{equation}

\begin{equation}
Event_{bias} = B = \sum _{m = 2} ^{\infty} P(m;\epsilon) \sim \epsilon ^{2} e ^{-\epsilon} / 2
\end{equation}

$B/U \sim \epsilon / 2$

\begin{equation}
P(m;\epsilon)P(any < DT) = \frac{\epsilon ^{m} e^{-\epsilon}}{m!}(1-exp(-(\frac{m!\delta}{2(m-2)!})))
\end{equation}

\begin{equation}
P(m;\epsilon)P(none < DT) = \frac{\epsilon ^{m} e^{-\epsilon}}{m!}exp(-(\frac{m!\delta}{2(m-2)!}))
\end{equation}

\begin{equation}
B/U = \frac{\sum _{2} ^{\infty} mP(m;\epsilon)P(any < DT)}{\sum _{1} ^{\infty} mP(m;\epsilon)P(none < DT)}
\end{equation}


\subsection{Iterative reconvolution}