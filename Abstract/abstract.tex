\abstract

In the field of cancer treatment, particle therapy has recently raised the interest of the community due to the high relative biological effectiveness of charged particles.
In particular they prove to be more effective for targeting radio resistant or inoperable tumours.
Nevertheless different sources of error can worsen the dose delivery profile, such as patient mis positioning, evolution of the tumour/morphology of the patient and imprecision in the treatment plannings, due to the fragmentation of the incident beam and range uncertainties.
As a consequence a three-dimensional non invasive imaging technique for ion beam therapy monitoring is required. The attention of the community is thus focused on positron emission tomography (PET).

In particular developments in the field of particle detectors push for the use of time of flight information, that allows to improve the sensitivity by improving the signal to noise ratio (SNR). 
In order to benefit from a significative reduction of the SNR, the target is less than 100 ps FWHM coincidence time resolution.
The standard solution for future PET scanners is to make use of heavy scintillating crystals coupled to Silicon PhotoMultipliers. 

This thesis is devoted to the full characterization of the parameters that influence time resolution in a scintillator/photodetector setup, with particular attention focused on the impact of time profiles of heavy scintillators on the performance.

The first part of the presented work has the objective of describing the fundamental model that governs light production and collection in a crystal.
To this purpose a model based on multi-exponential time profiles has been implemented on an existing framework, widening the scope of usage by evaluating the role of Cerenkov photons produced by low energy radiation.

Moreover in order to properly characterized the operational parameters of a scintillator setup, a comparative analysis of ray tracing softwares has been conducted, namely two packages SLitrani and Geant4. The latter has been chosen to build the simulation framework that allowed to disentangle the various source of resolution degradation.

Finally the work focused on the measurements and evaluation of rise time. Non zero rise time in scintillating systems is given by the different processes characterizing energy deposition inside a crystalline lattice, with utmost relevance of the latest stage of electron hole thermalization. The time scale of this phenomenon is $\sim$ 100 ps and until now has proven to be difficult to estimate due to the intrinsic limitations of detection setups.
Samples of different crystalline species such as CeF$_{3}$, LSO:Ce, LuAG:Ce, LuAG:Pr with different doping concentration are the subject of a time resolved analysis. 
Data taking has been performed in two different conditions: excitation at low energy (36 eV) and a PET-like setup (511 KeV).

The first set of measuremnts has been performed at the VUV beamline at Celia, Bordeaux, with an excitation energy of 36 eV. 
The data show results broadly separable in to two main groups: crystals in the LuAG group, with rise times $>$100 ps, and crystals belonging to the LSO group with rise times $<$50 ps. This is due to the different energy transfer mechanism.

The samples were then measured with a positron source ($^{22}$Na) on a experimental bench composed by a MCP-PMT stop detector and a tagging crystal readout by an amplified SiPM.
Rise time order of magnitude proved to be accessible, though mantaining large uncertainties due to the limited resolution and the long accumulation times.
Nonetheless we showed that Cerenkov photons and deep volume excitations introduce a non negligible contribution to the measured rise time. In particular the samples, excited above the Cerenkov threshold and in the deep volume of the crystal due to the energy of the excitation, showed slower rise times, above 100 ps.
Moreover the influence of Teflon diffusive wrapping have been investigated, showing that opening the extraction cone of the crystals leads to slower rise times due to coupling of multiple reflection modes.

%Nell'ambito oncologico terapeutico, la terapia adronica ha recentemente sollevato l'interesse della comunit\á in virt\ú dell'elevato effetto biologico (RBE) delle particelle cariche.
%In particolare si sono rivelate particolarmente efficaci per la cura di tumori radioresistenti o inoperabili.
%Tuttavia sono molteplici le sorgenti di degradazione del profilo di dose rilasciata: modifica nella posizione del paziente, evoluzione del tumore o dei tessuti, imprecisioni nel piano di trattamento dovute a incertezze nel range calcolato e frammentazione degli ioni pi\'u pesanti.
%Di conseguenza si rivela necessaria l'implementazione di un sistema di imaging tridimensionale non invasivo per il monitoraggio in vivo della terapia. L'attenzione della comunit\á si rivolge in particolare verso la Tomografia a Emissione di Positroni (PET).
%
%In particolare i pi\ú recenti sviluppo nell'ambito dei rivelatori di particelle ha indirizzato verso l'utilizzo della tecnologia time-of-flight (TOF), che consente di migliorare la sensitivit\á migliorando il rapporto segnale rumore (SNR). Al fine di beneficiare di una significativa riduzione del livello di rumore, l'obiettivo \é di raggiungere una risoluzione in coincidenza (CTR) inferiore ai 100 ps FWHM per i due rivelatori.
%La soluzione pi\ú comune per macchine PET vede l'impiego di scintillatori pesanti accoppiati a foto rivelatori, come i Silicon Photo Multiplier. 
%
%Questa tesi si concentra sulla caratterizzazione completa dei parametri che influenzano la risoluzione temporale in un sistema cristallo/foto rivelatore, con attenzione particolare rivolta all'impatto dei profili temporali di fluorescenza degli scintillatori sul rendimento del sistema.
%
%La prima parte del lavoro presentato si pone l'obiettivo di descrivere il modello fondamentale che governa la produzione e la raccolta di luce in un sistema PET. In questo senso \é stato ampliato un modello basato su una produzione di luce multi esponenziale, integrato dallo studio del ruolo dei fotoni Cerenkov nella formazione del segnale.
%
%Inoltre al fine di caratterizzare compiutamente i parametri operazionali di un rivelatore a cristallo scintillante, \é stata inizialmente svolta un'analisi comparativa di software di ray tracing, in particolare SLitrani e Geant4. Quest'ultimo \é stato scelto come software di riferimento per costruire il sistema di simulazioni che ha consentito una completa analisi dei dati sperimentali a dispoisizone.
%
%Infine il lavoro si \é concentrato sulla misura e determinazione del rise time. Un rise time non nullo in un sistema fluorescente \é dato dai differenti processi che caratterizzano la deposizione di energia nel reticolo cristallino, con particolare rivelanza dell'ultimo passaggio, la termalizzazione.
%La scala temporale di questo processo \é di 100 ps e finora si \é dimostrato particolarmente difficile da misurare a causa delle limitazioni sperimentali.
%Campioni di specie cristalline diverse (CeF$_{3}$, LSO:Ce, LuAG:Ce, LuAG:Pr) sono il soggetto di un'analisi ad alta risoluzione temporale. I dati sono stati raccolti in due condizioni di eccitazioni differenti: bassa energia (36 eV) e alta energia (511 KeV).
%
%Il primo gruppo di misure ha avuto luogo presso il fascio AURORE VUV nel centro di ricerca CELIA di Bordeaux, con un'eccitazione di 36 eV. I dati hanno mostrato risultati separabili in due famiglie differenti: per i cristalli appartenenti al gruppo dei Garnet (LuAG) il rise time risulta essere $>100$ ps, per le differenti variazioni del LSO  $<50$ ps. Questo \é dovuto al differente meccanismo di trasferimento di energia.
%
%I campioni sono quindi stati misurati con una sorgente di positroni (Na$^{22}$) su un banco sperimentale composto da un MCP-PMT come rivelatore di stop e un cristallo accoppiato a un SiPM come rivelatore di start.
%Il rise time si \é dimostrato accessibile a una misura sperimentale, mantenendo tuttavia larghe incertezze a causa della risoluzione limitata e dei lunghi tempi di accumulazione, che hanno limitato la raccolta statistica.
%Tuttavia abbiamo potuto mostrare che i fotoni Cerenkov e l'eccitazione nel volume del cristallo introducono una componente non trascurabile nel valore del rise time. In particolare, i campioni, eccitati sopra la soglia Cerenkov e nel volume, hanno mostrato rise time superiori a quelli registrati in VUV ($>100$ ps).
%Inoltre \é stata investigata l'influenza di riflettori diffusivi (e.g. Teflon), mostrando che a causa dell'ampliamento del cono ottico di estrazione dei cristalli il rise time aumenta significativamente.