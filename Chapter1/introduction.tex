\begin{savequote}[75mm] 
Nulla facilisi. In vel sem. Morbi id urna in diam dignissim feugiat. Proin molestie tortor eu velit. Aliquam erat volutpat. Nullam ultrices, diam tempus vulputate egestas, eros pede varius leo.
\qauthor{Quoteauthor Lastname} 
\end{savequote}

% COSA MANCA (bilbio, figure, formule ovviamente)
% radioterapia oggi:perche la si fa e quanto
% image reconstruction
% sensitivity
% outline

\chapter{The title of chapter one}

- radiotherapia oggi (vs surgery and chemio) - serve fonte

- serve fonte, ho usato katia parodi

Results of radiotherapy are improved when a high dose of radiation with high biological effectiveness is delivered to the tumour with the least possible dose to the surrounding tissues, especially in the case of critical organs \cite{Linz2011}.
In order to increase the conformity of the dose delivered to the tumour, diverse technologies have been considered and used.
Traditional forms of radiotherapy, X-ray tubes (energy $\sim$ 100 keV) or radioactive isotopes have been replaced by linear accelerators delivering $\sim$ 10 MeV from different directions (e.g. Intensity Modulated Radio Therapy), and provide treatment with photons and electrons.
Though being widely used as a standard in radiation therapy, the effectiveness of conventional electromagnetic radiation is limited by the intrinsic characteristics of interaction with matter.
In particular two aspects disfavours in principle the usability of electromagnetic radiation with respect to ion for tumour targeting: the depth dose profile, which does not allow for an optimal dose deposition to the tumour sparing vital organs
and the inferior biological effectiveness, which is the limiting factor in case of radio resistant tumours.
Heavier charged particles, like protons and ions (He-Ca) have the potentiality to overcome the limits of conventional therapy With respect to this, the scientific community is directing his attention towards possible improvements of ion beam therapy \cite{Amaldi2011}.

- qui citare entervision website

The work outlined in these pages have been sponsored by  The European training network in digital medical imaging for radiotherapy (ENTERVISION) at the European Center of Nuclear Research (CERN). ENTERVISION was established in February 2011 in response to the critical need for reinforcing research in online 3D digital imaging and the training of professionals in order to deliver some of the key elements and building blocks for realizing the vision for early detection and more precise treatment of tumours.

\section{Hadrontherapy}
\subsection{Ion beam therapy}

The first proposition of ion beam therapy was presented in 1946 by R. Wilson \cite{Wilson1946}. The original idea was exploit the physical properties of ion interaction in matter to improve the precision in radiotherapy treatments.  
Making use of the so called Bragg peak, that is using the fact that protons and ions in general deposit a maximum of energy at the end of their trajectory, the treatment could save the surrounding tissue from radiation overdose.
\begin{figure}  
\includegraphics[width=\textwidth]{figures/Chapter_1/conformal_x_rays}
\caption[Short figure name.]{Schematic view of depth-dose distributions of photons and ions. (a) photon
field, (b) spread-out ion beam, (c) depth–dose profiles along the central beam axis \cite{Linz2011}.}
\label{fig:myInlineFigure}}
\end{figure}

The dose deposited by photons, considered as the gold standard for tumour treatment, is maximum close to the beginning of the trajectory in the body and is characterized by an exponential decrease. As a consequence an undesired radiation dose is delivered to healthy tissues around the targeted tumour.

The recent therapeutic interest of ions in the field of radiotherapy relies mainly on their high relative biological effectiveness.
LET (linear energy transfer) has long been viewed as the main parameter to discern the biological effect of different kinds of radiation. It is a measure for the energy deposited by a charged particle traveling through matter. LET is closely related to stopping power and is not a constant value, since it changes along the particle's path (es 10 kEV/um for gamma, 100 kev/um for protons, 1000 kev/um for ions).
When considering ions of different atomic number LET becomes a limited parameter to evaluate the biological effect. In this sense the relative biological effectiveness (RBE) is considered the most accurate quantity, since it is defined as the biological effect of one type of ionizing radiation relative to another, given the same amount of absorbed energy. As the charge of the incident ions increases, so does the probability of severe DNA damage. An elevate RBE in the Bragg peak region has clearly been demonstrated for ions heavier than Helium \cite{Linz2011}.
As a consequence they prove to be more effective for targeting radio resistant or inoperable tumours.

\subsection{Beam delivery}

Ion beams are delivered by either cyclotrons or synchrotrons. In the first case the beam has a fixed energy which is tuned by means of degraders in order to deliver the correct dose profile. In the case of synchrotron the beam is delivered in spills and the energy is varied between spills. In the case of Carbon only synchrotrons can be used.
To deliver the dose to the planned target volume (PTV) different energies are superimposed in order to obtain the so-called spread-out Bragg peak (SOBP). The beam is usually delivered in a passive beam shaping setup or a scanning system. 

Different sources of error can worsen the dose delivery profile, such as patient mis positioning and evolution of the tumour/morphology of the patient. In addition the complex physics of ion interaction leads to  imprecision in the treatment plannings, due to fragmentation of the incident beam and range uncertainties.

Usually treatment planning systems cope with these problems by irradiating a volume larger than the tumour itself, called planning target volume (PTV) which contains the CTV. Complex compensating systems, including x-ray imaging techniques and patient positioning systems, allow to reduce errors in the dose profiles delivered. Treatment plannings of ion therapy relies for example on accurate values of particle range in tissue obtained from Hounsfield unit of computed tomograms, leading to uncertainties of $1-3\%$ in range calculations \cite{Enghardt2004}.
The dose delivered by a ion beam system is much more sensitive to these deviations than the one delivered by a photon beam. Due to the high biological effectiveness of ion beams wrong ranges could lead to dramatic under dosage to the tumour or over dosage to organ at risk surrounding.
As a consequence a three-dimensional non invasive imaging technique for ion beam therapy monitoring is required. Since ions, unlike photons, are stopped completely in the patient volume, technology like portal imaging are not suitable. The attention of the community is thus focused on positron emission tomography (PET), which relies on the peculiar characteristics of $\beta +$ decay.

\subsection{Monitoring of the beam}

Several attempts have already been undertaken to systematically assess the benefit of the PET method for beam monitoring, the principal one being the set up installed at the experimental carbon ion therapy unit at the Gesellschaft fur Schwerionenfroschung Darmstadt (GSI) \cite{Parodi2004}.
Two alternatives can be considered: the use of positron radioactive ions as projectiles for dose delivery or the detection of $\beta +$ activity given bi nuclei fragmentation.
As an example of the first approach it is interesting to consider the effort made at the Heavy Ion Accelerator in Chiba (Japan), where radioactive beams of $^{11}C - ^{10}C$ ions deliver an activity of $10^{3} - 10^{5}$ $Bq Gy^{-1} cm^{-3}$ within the irradiated volume.
Due to the low production rate of secondary radioactive ions, this approach has been only partially successful.
Another possibility is to make use of the $\beta +$ activation given by the fragmentation of stable ions interacting with the tissue.
The radioactivity is a direct product of the irradiation and, although the activity density is rather low (around 600 $Bq Gy^{-1} cm^{-3}$ for protons), this method provides a rather cheaper and feasible solution \cite{Enghardt2004}.
The activity slides very fast under a reasonable threshold for detectability and the most effective solution is an in-beam scanner.
In-beam PET is currently the main method implemented clinically for in situ monitoring of charged hadron radiotherapy\cite{Crespo2007}.

\section{Positron Emission Tomography}
\subsection{Principles}
Positron Emission Tomography (\ac{PET}) has been introduced as a nuclear medicine imaging technique which measures the distribution of a positron-emitting radionuclide (tracer), which is injected into the body on a biologically active molecule. In the case of in-beam PET the activity is present in the body of the patient due to the activation induced by proton interaction.
After the injection, or during the dose delivery, the subject of a \ac{PET} study is placed within the field of view (\ac{FOV}) of a number of detectors capable of registering incident gamma rays. The radionuclide in the radio tracer decays and the resulting positrons subsequently annihilate with electrons after travelling a short distance ($\sim$ 1 mm) within the body. In the case of in-beam PET activation of the tissues guarantees the detectability. Each annihilation produces two $511$ keV photons travelling in opposite directions and these photons may be detected by the detectors surrounding the subject. The detector electronics are linked so that two detection events unambiguously occurring within a certain time window may be called coincident and thus be determined to have come from the same annihilation. These "coincidence events" can be stored in arrays corresponding to projections through the patient and reconstructed using standard tomographic techniques. The resulting images show the tracer distribution throughout the body of the subject. The scheme of a PET scanner is shown in figure \ref{fig:scheme}.
Positron emission tomography relies on the $\beta ^{+}$ decay of a radionuclide.

The nucleus of the radionuclide can convert a proton into a neutron 
\begin{displaymath}
p\rightarrow n + e^{+} + \nu _{e}
\end{displaymath}
As positrons travel through human tissue, they give up their kinetic energy principally by Coulomb interactions with electrons. As the rest mass of the positron is the same as that of the electron, the positrons may undergo large deviations in direction with each Coulomb interaction, and they follow a tortuous path through the tissue as they give up their kinetic energy.

When the positrons reach thermal energies, they start to interact with electrons either by annihilation, which produces two $511$ keV anti-parallel photons, or by the formation of a hydrogen-like orbiting couple called positronium. In its ground-state, positronium has two forms: ortho-positronium, where the spins of the electron and positron are parallel, and para-positronium, where the spins are anti-parallel. Para-positronium again decays by self-annihilation, generating two anti-parallel $511$ keV photons. Ortho-positronium self-annihilates by the emission of three photons. Both forms are susceptible to the pick-off process, where the positron annihilates with another electron. Free annihilation and the pick-off process are responsible for over $80\%$ of the decay events.

\subsection{Image reconstruction}

\subsection{Sources of noise and sensitivity}
In a \ac{PET} scanner, each detector generates a timed pulse when it registers an incident photon. These pulses are then combined in coincidence circuitry, and if the pulses fall within a short time-window, they are deemed to be coincident (see figure).
A coincidence event is assigned to a line of response (\ac{LOR}) joining the two relevant detectors. In this way, positional information is gained from the detected radiation without the need of a physical collimator. This is known as electronic collimation.
When a physical collimator is used, directional information is gained by preventing photons which are not normal or nearly normal to the collimator face from falling on the detector. In electronic collimation, these photons may be detected and used as signal.
Coincidence events in \ac{PET} fall into four categories: true, scattered, random and multiple, as shown in figure . 

True coincidences occur when both photons from an annihilation event are detected by detectors in coincidence, neither photon undergoes any form of interaction prior to detection, and no other event is detected within the coincidence time-window.

A scattered coincidence is one in which at least one of the detected photons has undergone at least one Compton scattering event prior to detection. Since the direction of the photon is changed during the Compton scattering process, it is highly likely that the resulting coincidence event will be assigned to a wrong \ac{LOR}. Scattered coincidences add background to the true coincidence distribution which changes slowly with position, decreasing contrast and causing the isotope concentrations to be overestimated. They also add statistical noise to the signal. The number of scattered events detected depends on the volume and attenuation characteristics of the object being imaged, and on the geometry of the \ac{PET} scanner.

Random coincidences occur when two photons not arising from the same annihilation event are incident on the detectors within the coincidence time window of the system. The number of random coincidences in a given \ac{LOR} is closely linked to the rate of single events measured by the detectors joined by that \ac{LOR} and the rate of random coincidences increase roughly with the square of the activity in the \ac{FOV}. As with scattered events, the number of random coincidences detected also depends on the volume and attenuation characteristics of the object being imaged, and on the geometry of the scanner. The distribution of random coincidences is fairly uniform across the \ac{FOV}, and will cause isotope concentrations to be overestimated if not corrected for. Random coincidences also add statistical noise to the data.

% parte sulla sensitivity che manca

\subsection{TOFPET}

It has been shown that in-beam PET could not provide definitive information to the oncologist when medium to large tumors are involved\cite{Fiedler2006}. This is due to the operative parameters of scanners available on the market, with relatively slow scintillators and tomographs covering small solid angles. A decisive improvement could be given by time-of-flight PET (TOF-PET).

Recent developments in scintillator technology and read out electronics allow to build detectors able to detect the time difference between the moment of detection of the opposed gamma rays in coincidence. 

If we define a LOR between two detectors A and B, the distance between the center of the LOR and the annihilation point is given by

$x = (tb − ta ) · c/2$

where c is the speed of light.
Thus the spatial resolution is proportional to the coincidence time resolution (CTR) of the system.
Scanners available on the market today could deliver a 600 picoseconds time resolution, that translates to a positional uncertainties of 9 cm (FWHM) on the LOR.
The quality of the tomographic image largely benefits from the timing information of a TOFPET scanner, since it reduces considerably  the contribution of Compton scattered photons and from photons from outside the field-of-view (FOV). As a consequence the background from scattered and random coincidence is largely suppressed.
The signal to noise ration (SNR) is thus dramatically improved.
%qua sotto manca la voce in biblio
In the case of in-beam PET this is relevant, since it has been shown (Fiedler et al 2007) that during particle irradiation an considerable amount of activity is transported outside the FOV by metabolic processes. Moreover a high backround signal is typical of carbon ion beams (Pawelke 1997)

A useful and pratical estimation of the gain in signal to noise can be formalized as follows

$G = \frac{SNR_{TOF}}{SNR_{non_TOF}} = \sqrt{\frac{2\cdot D}{c \cdot CTR}}$

where D is the diameter of the volume under examination, c is the speed of light and CTR is the coincidence time resolution. Thus a CTR of 100 ps FWHM translates into a 1.5cm resolution on the position and a SNR gain of 5 (corresponing to a sensitivity gain of about a factor 25) compared to non TOF systems.

\section{Outline of the thesis}
The work presented in this thesis is devoted to the improvement of the operational parameters of TOFPET used in hadrontherapy, with particular respect to fast timing in particle detectors.
\section{From high energy physics to medical applications}

%mancano le citazioni

The present work was hosted by CERN, the European Organisation of Nuclear Research, based in Geneva, Switzerland. 
CERN was established by a formal act in Paris, on 1st of July 1953, as an organisation that ―shall provide for collaboration among European States in nuclear research of a pure scientific and fundamental character, and in research essentially related thereto -.
Thourought its history, CERN provided experimetnal and theoretical tools to study and understand the fundamental forces governing our universe, in a continous effort to improve our understanding of elementary  physics. 

The Large Hadron Collider (LHC), the last effort of scientific community around CERN facility, is the most powerful particle accelerator ever built, expected to accelerate proton beams to an energy of up to 7 TeV maintaining a luminosity of 1034 cm-2s-1. 
The work here presented has been hosted by the CERN section responsible for the development and construciont of the electromagnetic calorimeter (ECAL) of the CMS (Compact Muon Solenoid) experiment.
CMS is a multipurpose detectors used at LHC devoted to the identification and discoveries of new particles and evidence of physics beyond the standard model. It is composed by a 4 T solenoid magnet and several concentric subdetectors that include a tracker, an electromagnetic and a hadronic calorimeter and various layers of muon detectors. 
The ECAL detector was build with the fundamental contribution of the collaboration hosting this thesis: the Crystal Clear Collaboration. It was founded in 1990 as an international academic network of laboratories and industrial partners for the development of scintillating crystal detectors as well as their applications. It comprises experts in crystallography and solid state physics as well as in radiation detection and instrumentation. 
Its first goal was the development of a radiation-hard crystal for the ECAL detector, leading to the development of PbWO4 (PWO) as the material selected for CMS calorimeter. More recently the group has been focusing on the study of new materials for hadronic and electromagnetic calorimeters for future particle accelerators.
In parallel, the collaboration engaged in a effort of technology transfer to other domains exploiting the expertise developed in scintillating detectors. It is quite natural to focus the attention to medical physics, with particular respect to nuclear medicine. As shown in picture () the requirements for detectors used in medical physics and detectors for high energy experiments are similar.

\subsection{Study of time profiles}
- outline vero e proprio
	- panorama pubblicazioni
	- obiettivi
	- strumenti (simulazioni e setups)
	- misure

$x = 1/\alpha$ 

$$\zeta = \frac{1039}{\pi}$$

Fig.~\ref{fig:myFullPageFigure}.


\afterpage{\clearpage}


