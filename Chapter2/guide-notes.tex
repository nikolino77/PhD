\begin{savequote}[75mm] 
This is some random quote to start off the chapter.
\qauthor{Firstname lastname} 
\end{savequote}

% foto, quali cristalli in PET, rayleigh, Quenching, cit

\chapter{Scintillating detectors}

\section{Introduction to particle detectors}
In the field of medical applications, the energies of the gamma photons to be detected are usually of the order of hundreds of keV. In the case of PET scanners the energy of the two back to back photons is 511 keV.
A simple approach to estimate the parameters of the incoming radiation is to make use of a fluorescent sample coupled to a photodetector. A standard set up would include a heavy scintillator crystal which converts the incoming radiation into visible photons. The following steps of the detection process involve transportation to the entrance window of the photodetector, conversion of the photons into an electric signal and subsequent manipulation of the signal by readout electronics.  

% schema rivelatore

\section{Interaction of radiation with matter}
In this work we are mainly concerned with the interaction of gamma radiation with matter, thus focusing our attention to the three existing mechanisms: photo electric interaction, Compton interaction and pair production. A brief description of Rayleigh scattering will be given, as a non-ionizing type of scattering.
Moreover electrons produced by ionizing interactions can polarize the medium, giving origin to the Cerenkov effect and producing visible photons, which can be of foremost importance in the case of timing application.

\subsection{Photoelectric effect}

% schema fotoelettrico, sezione d urto

In the case of the photoelectric effect an electron from an atom is freed upon absorption of the incoming photon:

$\gamma + atom \rightarrow e^{-} + atom$

Due to conservation of momentum and energy this phenomenon does not occur with free electrons. 
The gamma energy trasnferred to the electron equals the binding energy of the electron itself minus its resulting kinetic energy $E_{eˆ{-}}$ 

$E_{e^{-}} = E_{\gamma} - E_{b}$

The photoelectric effect is predominant at low energies ($E<= 100 keV$) and favours tightly bound K-shell electrons. An approximation of the photo electric cross section is given by

%PROP
$\sigma _{pe} \prop \frac{Zˆ{n}}{E_{\gamma}ˆ{3.5}}$

The vacancy created can be filled through capture of bound or free electrons, eventually generating characteristics X-rays.  

\subsection{Compton scattering}

% schema Compton, sezione d urto, Klein Nishina

Compton scattering is the inelastic scattering of the incoming photon with a weakly bound electron in the material.

$\gamma + atom \rightarrow (\gamma ') + e^{-} + atom*$

Contrary to the photoelectric effect, this only concerns quasi-free electrons of the material. 
The photon trasnfers part of its energy to the electron, which is freed from its shell.
Applying conservation of energy and momentum it is possible to derive the energy of the scattered gamma as well as the direction and energy of free electron.

$E_{\gamma '} = \frac{E_{\gamma}}{1+\frac{E_{\gamma}}{m_{e}c^{2}}(1-cos\theta)}$

The angular distribution can be described by the Klein-Nishina formula. It is evident from the plot that forward scattering direction are favoured as the incoming photon energy increases

$\frac{d\sigma _{cpt}}{d\omega} = Z \cdot \frac{e^{2}}{4\pi \epsilon _{0} m_{e} c^{2}} \cdot \frac{1}{2} \cdot \frac{E'_{\gamma}}{E_{\gamma}} \right( 1 - \frac{E'_{\gamma}}{E_{\gamma}} \cdot sin^{2}\theta + \left[ \frac{E'_{\gamma}}{E_{\gamma}} ^{2} \right] \right)$

The total cross section can be computed by integrating the differential cross section over the angle.

% finire Compton con la cross section

\subsection{Pair production}

% pair production schema

If the energy of the gamma exceeds $2m_{e}c^{2} = 1.02 MeV$, the impinging photons can also be converted into an electron-positron pair. The cross-section of the pair production is given at low energies (thus low screening) by

$\sigma _{pair} = 4\alpha r_{e}^{2} Z^{2} \left( \frac{7}{9}ln2\frac{E}{m_{e}c^{2}} - \frac{109}{54}\right)$

The cross section is very low compared to that of photoelectric and Compton effect until the energy of the gamma approaches several electron Volts. Thus for the energies involved in medical applications pair production can be neglected.

\subsection{Rayleigh scattering}

% questo e la relative importance dei fenomeni


\section{The scintillation mechanism}
% figura bande semplice
As a general idea the scintillation process can be considered as the conversion of the energy of an incident gamma quantum or particle into a certain number of low energy photons\cite{Rodnyi1997}. In a way it can be therefore defined as a wavelentgh shifting process\cite{Lecoq2006}.

After a ionization event, generated by the mechanisms presented above in the case of a gamma interaction, the scintillator relaxes towards a new equilibrium. This process is characterized by a multitude of sub processes, that can be depicted by band diagrams as the one in picture.
As long as the energy of the particles is high enough, it is transferred to secondary particles of low energy, creating an electromagnetic cascade.
A crystal though is an ordered ensemble of atoms, so the electrons in the keV range start to couple with electrons and atoms of the lattice. As a result of their interaction with electronic states of the material, couples of electrons and relative vacancies are created. The electron hole pairs migrate in the lattice above and below the ionization threshold until they are trapped by a defect or recombine on a luminescent center. Alternatively they cool down by coupling to the lattice vibrations until they reach the top of the valence band (hole) or the bottom of the conduction band (electron). They can also form loosely bound structures called exciton, with en energy slightly smaller than the bandgap energy.
The scintillator itself must contain luminescent centers, either intrinsic or extrinsic (doping ions). These molecular systems in the lattice present characteristic transitions between excited states.
The scintillation process can therefore be represented as the sequence of the following stages\cite{Rodnyi1997}: 

\begin{itemize}
\item Absorption of ionizing radiation and creation of primary e-h pairs
\item Relaxation of primary e-h pairs with production of secondary e-h pairs, plasmons, photons, etc.
\item Thermalization of low energy e-h pairs down to the band gap energy $E_{g}$
\item Energy transfer form the e-h pairs to the luminescence centers
\item Emission of scintillation photons
\end{itemize}

\subsection{Creation of electron hole pairs}

To analyze more in depth the mechanisms of the scintillation, we can consider an intermediate energy gamma ray ( $\prop 500 keV$) interacting with the scintillator material. In this case the photoelectric effect is dominant. Thus it will produce a hole in a inner shell (usually K shell) and a free or quasifree electron.

$A + h\ni \rightarrow A^{+} + e$

The energy of the primary electron will be $h\ni - E_{k}$ where $E_{k}$ is the K level energy.

The relaxation then happens differently for electrons and holes. 

The ionized atom ($A^{+}$) can relax either radiatively, thus emitting a photon, or nonradiatively, generating a secondary electron. This is know as the Auger effect. Thereafter a cascade of both radiative and nonradiative processes take place.
The Auger electron and the primary electron begin a proces of electron-electron scattering or phonon emission. In the case of a radiative emission, the soft x-ray photon emitted may be absorbed producing a new deep hole and free electron. 

The electron on the other hand will ionize an atom

$A + e \rightarrow A^{+} + 2e$

The two undistiguishable electrons will undergo a number of other ionization processes, resulting in an avalanche of secondary electrons and holes. At some point the secondary products of these processes are not able to ionize the medium anymore.
A fast electron can in principle interact also with valence electrons of the medium, producing collective oscillations known as plasmons. Plasmons behave as quasiparticles, with an energy of $\sym 10 eV$ and can decay into e-h pairs.

This ensemble of avalanche processes continues until the generated secondaries are not able to create further ionization. At this point electrons and holes start to interact with the vibrations of the lattice in a stage called thermalization, via different mechanisms of electron-phonon interaction. 
 As a consequence, at the end of this chain of de-excitation processes, low energy electronic excitations are present: electrons in the conduction band, holes in the valence band, valence excitons, core excitons.
 
\subsection{Intrinsic luminescence}

Electron and holes have several ways to recombine after thermalization and give rise to scintillation photons.
The simplest emission process is direct recombination

$e + h \rightarrow h\ni$

Recombination can more effectively take place when the energy of the electron and hole has decreased, so that they form a loosely bound structure called exciton. 
However the various impurities and lattice defects play a very important role in the scintillation process. Thermalized carriers can be bound in some places of the lattice where atom or defects are localized. 
For example many ionic crystals shows phenomena of localization of the valence hole in the lattice, known as self-trapping. This structure appears when a thermalized hole localizes an anion, polarizing the environment. As a result the hole can be shared between two neighbouring ions forming a $V_{k}$ center, and the hole is defined as self-trapped hole. For high energy excitation direct creation of valence exciton is unlikely, so $V_{k}$ centers usually capture free electrons. From subsequent de excitation they can emit photons, thus giving rise to the excitonic luminescence.
 
$e + h \rightarrow ex \rightarrow h\ni$

\subsection{Core to valence transitions}

If the core bands of the scintillator lie below the Ager threshold, the most favoured transitions involve holes in the valence band and electron in the conduction band. Some systems though present the so-called cross luminescence. This phenomenon implies a direct core to valence transition, due to the fact that holes in uppermost core bands can not deexcite non radiatively\cite{Lecoq2006}. 

A notable example of core to valence transition is $BaF_{2}$. In this system a $Ba^{2+}$ $5p$ core hole is above the Auger threshold and hence Auger effect does not occur. They can recombine directly with electrons from the valence band, in most of the cases radiatively.
This leads to a very fast luminescence given by recombination of the core hole, while the primary electron de excitation is more complex thus leading to a slower component.

%immagine 

\subsection{Extrinsic luminescence}

The scintillator samples used in this work are extrinsic, that is doped with activation centers that can enhance the intrinsic scintillation properties presented above by favouring direct recombiantion.
Rare earth ions doping, for example, is largely used in scintillator technology because of the parity and spin-allowed transition $4fˆ{n-1}5d\rightarrow 4fˆ{n}$. 
Extrinsic scintillators usually present different luminscent mechanisms driven by activated sites\cite{Lecoq2006}:
\begin{itemize}
\item $e + h + A \rightarrow ex + A \rightarrow A* \rightarrow A + h\ni$
\item $e + h + A \rightarrow A^{1+} + e \rightarrow A* \rightarrow A + h\ni$
\item $e + h + A \rightarrow (A^{1-})* + h \rightarrow A + h\ni$
\item $A \rightarrow A* \rightarrow A + h\ni$
\end{itemize}
In the first case the insertion of dopants is able to sufficiently quench the exciton luminescence so that excitation of radiative centers results form a transfer from excited matrix states.
A copeting process is the direct capture of free thermalized carriers by luminescent center, in the case of electrons or holes.
In heavy doped or self-activated crystals ($CeF_{3}$) direct excitation by ionizing radiation is possible.

%tabella e aggiungi i riferimenti alle formule nel testo

\section{Quenching phenomena}

% thermal, concentration, trapping Chapter3 Lecoq

\subsection{Light yield}
One of the feature commonly required of a scintillator is to be have a high light yield, that is to be an efficient converter of radiation to visible light.
In this case the relative light output of the scintillator, $L_{R}$, can be considered the significant quantity. It is defined as the number of emitted photons per unit of absorbed energy\cite{Rodnyi1997}

$L_{R} = \frac{N_{ph}}{E_{\gamma}}$

The number of produced e-h pairs $N_{eh}$ depends on the average energy needed for the creation of a low energy e-h pair, $\chi _{eh}$. This value depends on the type of lattice and band gap of the material, with a numericl coefficient $\beta$

$\chi _{eh} = \beta \cdot \ E_{g}$

If $\alpha$ is the average number of scintillation photons produced by a single e-h pair, the light output is

$L_{r} = \frac{\alpha \cdot N_{eh}}{E_{\gamma}} = \frac{\alpha}{\chi _{eh}} = \frac{\alpha}{\beta \cdot \E_{g}}$

The coefficient $\alpha$ depends on the transport efficiency of the e-h pairs to the luminescence center and the conversion efficiency of the center itself.

\subsection{Optical properties and light transport}
%The relative light yield light transport and absorption
Moreover it is clear that the emission band of a scintillator should lie in the spectral range of the optical transmission of the crystal.

\subsection{Energy resolution and nonproportionality}
In the case of gamma spectroscopy ut is necessary to discriminate quanta with different energy.
For scintillation detector this fundamental property is characterized by the energy resolution $R$, defined as $\Delta /E$ (in $\%$) where $\Delta E$ is the full width at half maximum (FWHM) at pulse height $E$.
It depends on the characteristics of the scintillator, i.e. materials, size and defects as well as the coupling with the photo detector and the parameters of the photo detectors itself. Statistical fluctuations at any step of the detector chain, from dynode multiplication from photocathode efificency in the case of a PMT can worsen the resolution at the peak. Thus energy resolution can be defined as\cite{Rodnyi1997}

$R^{2} = R_{S}^{2} + R_{PM}^{2} = R_{S}^{2} + \frac{\delta}{E_{\gamma}}$

where $R_{S}$ and $R_{PM}$ are, respectively, the scintillator and photomultiplier contributions and $\delta$ includes photo electron statistics.
It is possible to further decompose the scintillator resolution $R_{S}$ to take into account the factors depending on the type of scintillator used. In particular it is useful to introduce a term for the transfer efficiency of the optical photons $R_{S}$, a term for inhomogeneity $R_{i}$ and a term for nonproportionality $R_{n}$

$R_{S}^{2} = R_{t}^{2} + R_{i}^{2} + R_{n}^{2}$

The interest lies in the fact that the two terms, for inhomogeneity and nonproportionality, account for the intrinsic resolution of the crystal.
Inhomogeneity arise from possible imperfections of the scintillator, such as local variations in the concentration of the dopant or optical defects.
Non proportionality arise when scintillators show deviation from stability of excitation spectrum, that is when linearity between energy of the excitation and relative light output is not preserved. This is particularly important for low energy excitation, since scintillation phenomena occur mainly on the surface. Non proportionality is a cause of the statistical nature of the creation of secondary electrons and photons and contribute to worsen the resolution.

\subsection{Cerenkov effect}
The effect of Cerenkov radiation brings important information both in high energy physics and time resolved PET.
Cerenkov radiation occurs when a charged particle passes through a dieletric medium at a speed greater than the phase velocity of light in that medium.
The phase velocity of light in a medium of refractive index $n > 1$ is

$v_{light} = \frac{c}{n}$

A charged particle can travel faster than the speed of light if, given its velocity $v_{p}$ 

$\frac{c}{n} < v_{p} < c$

This translates to the following condition for the $\beta$ coefficient of the particle

$\beta = \frac{v}{c} > \frac{1}{n}$

For a particle of a given mass thus the energy threshold is

$K_thr = mc^{2}\left( \frac{\sqrt {n^{2}-1}}{n} - 1 \right)$

The phenomenology of Cerenkov effect can be explained considering the polarization of the medium caused by a charged particle trasversing it.
Below the Cerenkov threshold the dipoles sorrounding are simmetrically arranged around the path. As the particle crosses the threshold it travels faster the the speed at which it interacts with the dipoles. This simmetry breaking leads to a non-vanishing dipole moment and thus to the formation of a wave front.

Cerenkov photons are emitted at a characteristic angle in the forward direction, obtained via simple geometrical considerations. The distance traveled by the charged particle in a time t is $t\cdot \beta \cdot c$ whereas the distance along which the photon propagates is $t\cdot c /n$ as shwn in fig.

Therefore the characteristic angle at which photons are emitted can be calculated as

$cos(\teta _{C}) = frac{t c/n}{t \beta c} = \frac{1}{n\beta}$

As will be shown in the next chapter, the direction of emission retains a primary interest in the field of particle identification, while it has little impact on timing measurement on PET scanners. It is worth to be noted though that the Cerenkov photons are emitted promptly, taking a relevant share of the first incoming photons.  

It is useful to consider the number of emitted photons per unit length by a charged particle in function of the wavelength

$\frac{d^N}{d\lambda dx} = \frac{2\pi z^{2}\alpha}{\lambda ^{2}}\left( 1 - \frac{1}{\beta ^{2}n^{2}(\lambda)} \right)$

Neglecting dispersion in the medium, and integrating over an appropriate interval of waveleghts we obtain that the photons are emitted mostly in the UV range.

$\frac{dN}{dx} = 2\pi z^{2} \alpha \left( 1-\frac{1}{\beta ^{2} n^{2} (\lambda)}\right) \int _{\lambda _{1}} ^{\lambda _{2}} \frac{d\lambda}{lambda ^{2}}  = 2\pi z^{2}\alpha sin^{2}\theta _{C} \left( \frac{1}{\lambda _{1}}-\frac{1}{\lambda _{2}}\right)$

\section{Scintillators in PET detectors}

% pensarci un po
% cmq parametri light yield energy resolution density velocita

% accennare che la issue del timing si vede nel capitolo dopo

% quelli che si usano di piu e perche
